%% LyX 2.1.3 created this file.  For more info, see http://www.lyx.org/.
%% Do not edit unless you really know what you are doing.
\documentclass[12pt,a4paper,twoside,english,BCOR7.5mm]{article}
\usepackage[T1]{fontenc}
\usepackage[latin9]{inputenc}
\usepackage{color}
\usepackage{babel}
\usepackage{url}
\usepackage{amsmath}
\usepackage[all]{xy}
\usepackage[unicode=true,
 bookmarks=true,bookmarksnumbered=true,bookmarksopen=true,bookmarksopenlevel=2,
 breaklinks=false,pdfborder={0 0 1},backref=false,colorlinks=true]
 {hyperref}
\hypersetup{pdftitle={Using XY-pc in LyX},
 pdfauthor={H. Peter Gumm},
 pdfsubject={LyX's XY-pic manual},
 pdfkeywords={LyX, documentation},
 linkcolor=black, citecolor=black, urlcolor=blue, filecolor=blue,pdfpagelayout=OneColumn, pdfnewwindow=true,pdfstartview=XYZ, plainpages=false}

\makeatletter

%%%%%%%%%%%%%%%%%%%%%%%%%%%%%% LyX specific LaTeX commands.
\pdfpageheight\paperheight
\pdfpagewidth\paperwidth


%%%%%%%%%%%%%%%%%%%%%%%%%%%%%% User specified LaTeX commands.
% DO NOT ALTER THIS PREAMBLE!!!
%
% This preamble is designed to ensure that the file prints
% out as advertised. If you mess with this preamble,
% parts of this document may not print out as expected.  If you
% have problems LaTeXing this file, please contact 
% the documentation team
% email: lyx-docs@lists.lyx.org

\usepackage[all]{xy}

% define new commands used in sec. 5.1
\newcommand{\xyR}[1]{
  \xydef@\xymatrixrowsep@{#1}}
\newcommand{\xyC}[1]{
  \xydef@\xymatrixcolsep@{#1}}

\newdir{|>}{!/4.5pt/@{|}*:(1,-.2)@^{>}*:(1,+.2)@_{>}}

% a pdf-bookmark for the TOC is added
\let\myTOC\tableofcontents
\renewcommand\tableofcontents{%
  \pdfbookmark[1]{\contentsname}{}
  \myTOC }

% redefine the \LyX macro for PDF bookmarks
\def\LyX{\texorpdfstring{%
  L\kern-.1667em\lower.25em\hbox{Y}\kern-.125emX\@}
  {LyX}}

% if pdflatex is used
\usepackage{ifpdf}
\ifpdf

% set fonts for nicer pdf view
\IfFileExists{lmodern.sty}
 {\usepackage{lmodern}}{}

\usepackage{coqdoc}
\usepackage{amsmath,amssymb}
\usepackage{url}
\usepackage{makeidx,hyperref}

\fi % end if pdflatex is used

\makeatother

\begin{document}

\title{Maclane pentagon is some comonadic descent (Rough Proof)}


\author{Christopher Mary\\
EGITOR.NET, https://github.com/mozert}
\maketitle
\begin{abstract}
This text responds to Gross \emph{Coq Categories Experience} \cite{gross}
and Chlipala \emph{Compositional Computational Reflection} \cite{chlipala}
of ITP 2014. ``Compositional'' is synonymous for functional/functorial;
``Computational Reflection'' is synonymous for monadic semantics;
and this text attempts some comonadic descent along functorial semantics
: Dosen semiassociative coherence covers Maclane associative coherence
\cite{dosen}.
\end{abstract}

\section{Contents }

This text responds to Gross \emph{Coq Categories Experience} \cite{gross}
and Chlipala \emph{Compositional Computational Reflection} \cite{chlipala}
of ITP 2014. ``Compositional'' is synonymous for functional/functorial;
``Computational Reflection'' is synonymous for monadic semantics;
and this text attempts some comonadic descent along functorial semantics
: Dosen semiassociative coherence covers Maclane associative coherence
\cite{dosen}.

Categories \cite{borceuxjanelidze} \cite{borceux} study the interaction
between reflections and limits. The basic configuration for reflections
is : 
\[
\xyC{5pc}\xyR{5pc}\xymatrix{\text{CoMod}\ar[r]^{\text{{reflection}}}\ar[d]_{\text{{identity}}}\ar@{}[ddr]\sp(0.2){\underset{\Longrightarrow}{\text{unit}}} & \text{Mod}\ar[dl]^{\text{{coreflection}}}\\
\text{CoMod}\ar[d]_{\text{{reification}}} & \ \\
\text{Modos} & \ 
}
\]
 where, for all $\text{reification}$ functor into any $\text{Modos}$
category, the map\\
 $(\ \_\star\text{reflection})\circ(\text{reification}\star\text{unit})$
is bijective, or same, for all object $M'$ in $\text{CoMod}$, the
polymorphic in $M$ map\\
 $(\text{coreflection}\ \_\ )\circ\text{unit}_{M'}:\text{Mod}(\text{reflection }M',M)\rightarrow\text{CoMod}(M',\text{coreflection }M)$
is bijective (and therefore also polymorphic in $M'$ with reverse
map $\text{counit}_{M}\circ(\text{reflection }\_\ )$ whose reversal
is polymorphically determined by $(\text{coreflection}\star\text{counit})\circ(\text{unit}\star\text{coreflection})=\text{identity}$
and $(\text{counit}\star\text{reflection})\circ(\text{reflection}\star\text{unit})=\text{identity}$
); and it is said that the $\text{unit}$ natural/polymorphic/commuting
transformation is the unit of the reflection and the reflective pair
$(\text{reification}\circ\text{coreflection},\text{reification}\star\text{unit})$
is some coreflective (``Kan'') extension functor of the $\text{reification}$
functor along the $\text{reflection}$ functor. 


\section{Misc}

MACLANE PENTAGON IS SOME COMONADIC DESCENT. LOGICAL COHERENCE. CATEGORICAL
DESCENT. COQ DEDUKTI MODULO.

1. LOGICAL COHERENCE -GROSS, CHLIPALA, COMONADIC FUNCTORIAL SEMANTIC,\textasciitilde{}\textasciitilde{}
GALOIS, GROTENDIECK, GENTZEN -ERRORS OF DOSEN 1 CONFUSE MIXUP CONVERTIBLE
(DEFINITIONALLY/META EQUAL) WITH PROPOSITIONAL EQUAL WHEN THE THINGS
ARE VARIABLES INSTEAD OF FULLY CONSTRUCTOR-ED EXPLICIT TERMS 2. RELATED
TO THIS IS THE COMPUTATIONNALLY WRONG ORDER OF HES PRESENTATION -METAAUTO
LOGICAL PROGRAMMING CONTRAST REFLECTION PROGRAMMING -CATEGORICAL LOGICAL
COHERENCE AS META REFLECTION PROGRAMMING -induction; (eval beta iota;
auto logical unify; auto substitution rewrite); repeat ( (match goal
| context match term => destruct); (eval beta iota; auto logical unify;
auto substitution rewrite) ); congruence; omega; anyreflection -convert
or logify recur or unify or substitute rewrite or reflect

(... classification more than property specification subset, example:
regurlar expression, red-black tree, closed categories, categories
... ... recur on applied subterm or nested subterm ...)

2. CATEGORICAL DESCENT -POLYMORPHIC REFLECTION (ADJUNCTION)\textasciitilde{}\textasciitilde{},
MONADIC ADJUNCTION -INTERNALISATION -> FUNCTORIAL SATURATION (YONEDA
FREE ALGEBRA) -TENSOR OF THEORIES -COMONADICITY SHALL BE VERY RELATED
TO COHERENCE -KAN EXTENSIONS IS CONTEXT FOR PROOFS BY REIFICATION/REFLECTION
-BASE FOR KAN EXTENSIONS MAYBE MODOS, MOVE FROM CARTESIAN LIMITS TO
?? -(NORMALIZE\_ARROW\_ASSOC ON NORMAL IS REVERSIBLE THEREFORE NORMALIZE\_UNIT\_ASSOC
IS UNIT OF REFLECTION) -SemiAssoc <--> Assoc COMONADIC , NO LACK NEWMAN
CONFLUENCE List <--> SemiAssoc MONADIC , LACK NEWMAN CONFLUENCE, MAYBE
\textquotedbl{}ITERCOVER RESOLUTION\textquotedbl{} ? -MORE BY INTERNALISATION
OF COMMON COHERENCES -NOT YET FULL CATEGORICAL DESCENT FORM, PROGRAMME
LOGICAL DOSEN AND CATEGORICAL BORCEUX 1 2 AUTO DESCENT VIEW, RELATE
CONVERTIBILITY TYPES BY AUTORESOLUTION OR TACTIC, RELATE CANONICAL-STRUCTURE-AND-COERCION
RESOLUTION

3. COQ DEDUKTI MODULO -COQTEXT ASSOC RECURSIVE SQUARE, NO ASSOC PREORDER
-COQTEXT SEMIASSOC COMPLETENESS, NO SEMIASSOC CONFLUENCE, NO SEMIASSOC
PREORDER 

\include{C:/Users/User/Documents/user1/tmp/hohocaml/itp2015/coherence2.v}
\begin{thebibliography}{1}
\bibitem{gross}Jason Gross, Adam Chlipala, David I. Spivak. \textquotedblleft Experience
Implementing a Performant Category-Theory Library in Coq\textquotedblright .
In: Interactive Theorem Proving. Springer, 2014.

\bibitem{chlipala}Gregory Malecha, Adam Chlipala, Thomas Braibant.
\textquotedblleft Compositional Computational Reflection\textquotedblright .
In: Interactive Theorem Proving. Springer, 2014.

\bibitem{dosen}Kosta Dosen, Zoran Petric. \textquotedblleft Proof-Theoretical
Coherence\textquotedblright . \\
\url{http://www.mi.sanu.ac.rs/~kosta/coh.pdf} , 2007.

\bibitem{borceuxjanelidze}Francis Borceux, George Janelidze. \textquotedblleft Handbook
of categorical algebra. Volumes 1 2 3.\textquotedblright . Cambridge
University Press, 2001.

\bibitem{borceux}Francis Borceux. \textquotedblleft Handbook of categorical
algebra. Volumes 1 2 3.\textquotedblright . Cambridge University Press,
1994.\end{thebibliography}

\end{document}
