%% LyX 2.0.6 created this file.  For more info, see http://www.lyx.org/.
%% Do not edit unless you really know what you are doing.
\documentclass[12pt,letterpaper,english,BCOR7.5mm]{amsart}
\usepackage[T1]{fontenc}
\usepackage[utf8x]{inputenc}
\usepackage{color}
\usepackage{babel}
\usepackage{url}
\usepackage{amsthm}
\usepackage{amstext}
\usepackage{amssymb}
\usepackage[all]{xy}
\usepackage[unicode=true,
 bookmarks=true,bookmarksnumbered=true,bookmarksopen=true,bookmarksopenlevel=2,
 breaklinks=false,pdfborder={0 0 1},backref=false,colorlinks=true]
 {hyperref}
\hypersetup{pdftitle={Maclane pentagon is some comonadic descent},
 pdfauthor={Christopher Mary},
 linkcolor=black, citecolor=black, urlcolor=blue, filecolor=blue,pdfpagelayout=OneColumn, pdfnewwindow=true,pdfstartview=XYZ, plainpages=false}
\usepackage{breakurl}

\makeatletter

%%%%%%%%%%%%%%%%%%%%%%%%%%%%%% LyX specific LaTeX commands.
\special{papersize=\the\paperwidth,\the\paperheight}


%%%%%%%%%%%%%%%%%%%%%%%%%%%%%% Textclass specific LaTeX commands.
\numberwithin{equation}{section}
\numberwithin{figure}{section}

\@ifundefined{date}{}{\date{}}
%%%%%%%%%%%%%%%%%%%%%%%%%%%%%% User specified LaTeX commands.
% DO NOT ALTER THIS PREAMBLE!!!
%
% This preamble is designed to ensure that the file prints
% out as advertised. If you mess with this preamble,
% parts of this document may not print out as expected.  If you
% have problems LaTeXing this file, please contact 
% the documentation team
% email: lyx-docs@lists.lyx.org

\usepackage[all]{xy}

% define new commands used in sec. 5.1
\newcommand{\xyR}[1]{
  \xydef@\xymatrixrowsep@{#1}}
\newcommand{\xyC}[1]{
  \xydef@\xymatrixcolsep@{#1}}

\newdir{|>}{!/4.5pt/@{|}*:(1,-.2)@^{>}*:(1,+.2)@_{>}}

% a pdf-bookmark for the TOC is added
\let\myTOC\tableofcontents
\renewcommand\tableofcontents{%
  \pdfbookmark[1]{\contentsname}{}
  \myTOC }

% redefine the \LyX macro for PDF bookmarks
\def\LyX{\texorpdfstring{%
  L\kern-.1667em\lower.25em\hbox{Y}\kern-.125emX\@}
  {LyX}}

% if pdflatex is used
\usepackage{ifpdf}
\ifpdf

% set fonts for nicer pdf view
\IfFileExists{lmodern.sty}
 {\usepackage{lmodern}}{}

\fi % end if pdflatex is used

\usepackage[utf8x]{inputenc}
\usepackage[T1]{fontenc}
\usepackage{lmodern}
\usepackage{fullpage}
\usepackage{coqdoc}
\usepackage{amsmath,amssymb}
\usepackage{url}
 \usepackage{hyperref}
% \usepackage{makeidx,hyperref}

\makeatother

\begin{document}

\title{Maclane pentagon is some comonadic descent}


\author{Christopher Mary\\
EGITOR.NET, https://github.com/mozert}
\begin{abstract}
\textsc{Rough Proof}.\emph{ }This Coq text responds to Gross \emph{Coq
Categories Experience} \cite{gross} and Chlipala \emph{Compositional
Computational Reflection} \cite{chlipala} of ITP 2014. ``Compositional''
is synonymous for functional/functorial and ``Computational Reflection''
is synonymous for monadic semantics; and this text attempts some comonadic
descent along functorial semantics : Dosen semiassociative coherence
covers Maclane associative coherence by some comonadic adjunction,
{$\text{embedding }:\ \text{SemiAssoc}\leftrightarrows\text{Assoc}\ :\text{ flattening reflection}$.
}
\end{abstract}
\maketitle

\section{Contents }

Categories \cite{borceux} study the interaction between reflections
and limits. The basic configuration for reflections is : 
\[
\xyC{5pc}\xyR{5pc}\xymatrix{\text{CoMod}\ar[r]^{\text{{reflection}}}\ar[d]_{\text{{identity}}}\ar@{}[ddr]\sp(0.2){\underset{\Longrightarrow}{\text{unit}}} & \text{Mod}\ar[dl]^{\text{{coreflection}}}\\
\text{CoMod}\ar[d]_{\text{{reification}}} & \ \\
\text{Modos} & \ 
}
\]
 where, for all $\text{reification}$ functor into any $\text{Modos}$
category, the map 
\[
(\ \_\star\text{reflection})\circ(\text{reification}\star\text{unit})
\]
 is bijective; or same, for all object $M'$ in $\text{CoMod}$, the
polymorphic in $M$ map { 
\[
(\text{coreflection}\ \_\ )\circ\text{unit}_{M'}:\text{Mod}(\text{reflection }M',M)\rightarrow\text{CoMod}(M',\text{coreflection }M)
\]
 }is bijective, and therefore also polymorphic in $M'$ with reverse
map $\text{counit}_{M}\circ(\text{reflection }\_\ )$ whose reversal
equations is polymorphically determined by 
\begin{eqnarray*}
(\text{coreflection}\star\text{counit})\circ(\text{unit}\star\text{coreflection}) & = & \text{identity}\\
(\text{counit}\star\text{reflection})\circ(\text{reflection}\star\text{unit}) & = & \text{identity\ }.
\end{eqnarray*}
 And it is said that the $\text{unit}$ natural/polymorphic/commuting
transformation is the \emph{unit of the reflection} and the reflective
pair $(\text{reification}\circ\text{coreflection},\text{reification}\star\text{unit})$
is some \emph{coreflective (``Kan'') extension functor} of the $\text{reification}$
functor along the $\text{reflection}$ functor. This text shows some
comonadic adjunction, { $\text{embedding }:\ \text{SemiAssoc}\leftrightarrows\text{Assoc}\ :\text{ flattening reflection}$.
}

Categories \cite{borceuxjanelidze} \cite{borceux} converge to the
\emph{descent technique}, this convergence is from both the \emph{functorial
semantics technique} with the \emph{monadic adjunctions technique}.
Now functorial semantics starts when one attempts to \emph{internalize}
the common phrasing of the logician model semantics, and this internalization
has as consequence some functionalization/functorialization saturation/normalization
of the original theory into some more synthetic theory; note that
here the congruence saturation is some instance of postfix function
composition and the substitution saturation is some instance of prefix
function composition. The ``Yoneda''/normalization lemma takes its
sense here. And among all the relations between synthetic theories,
get the \emph{tensor of theories}, which is some \emph{extension of
theories}, and which is the coproduct (disjoint union) of all the
operations of the component theories quotiented by extra \emph{commutativity}
between any two operations from any two distinct component theories;
for example the tensor of two rings with units as synthetic theories
gives the bimodules as functorial models.

Now Galois says that any radical extension of all the \emph{symmetric
functions} in some indeterminates, which also contains those indeterminates,
is abe to be incrementally/resolvably saturated/``algebra'' as some
further radical extension whose interesting endomorphisms include
all the permutations of the indeterminates. And when there are many
indeterminates, then some of those permutations are properly preserved
down the resolution ... but the resolution vanish any permutation
! In this context of saturated extensions, one then views any polynomial
instead as its quotient/ideal of some ring of polynomials and then
pastes such quotients into ``algebraic algebras'' or ``spectrums''
or ``schemes'' .. This is Galois descent along Borceux-Janelidze-Tholen
\cite{borceuxjanelidze}

\emph{Esquisse d'un programme} : The raw combinatorial (``permutation
group'') angle converge to Aigner \cite{aigner}. Another parallel
of the raw combinatorial techniques of Galois is the raw proof techniques
of Gentzen that inductive recursive arithmetic cannot well-order some
ordinal. One question is whether the descent techniques and the proof
techniques can converge. The initial item shall be to internalize/functorialize
the semantics of Dosen book and do \emph{automation programmation}
so to gather data and examples and experiments. The automation programming
technique has one common form mixing induction or simplification conversion
or logical unification or substitution rewriting or repeated heuristic/attempt
destructions or reflective decision procedure; for example the form
behind \emph{crush} of Chlipala CPDT \cite{chlipalacpdt} is :
\begin{quote}
{[} induction; \\
(eval beta iota; auto logical unify; auto substitution rewrite); \\
repeat ( (match goal | context match term => destruct); \\
(eval beta iota; auto logical unify; auto substitution rewrite) );
\\
congruence; omega; anyreflection {]}
\end{quote}
The next item shall be to memo Borceux books 1 and 2, not only from
the simplifying conversion angle or the unification (``logic programming'')
angle or the substitution rewrite angle, but also from the computational
reflection angle. This computational reflection angle shall be far
more than decision procedures, but rather shall be descent techniques
for existence (fullness) and identification (failfulness); so this
would allow for implicit arguments to be resolved after descent or
for some arguments to be programmed after descent to some easier terminology
.. This is 3 pages/day = 1 year memo reading.\begin{coqdoccode}
\coqdocemptyline
\coqdocemptyline
\end{coqdoccode}
\section{Maclane Associative Coherence}

\begin{coqdoccode}
\coqdocemptyline
\coqdocemptyline
\end{coqdoccode}
This \textsc{Coq} text shows the semiassociativity completeness and coherence internal some encoding where associative coherence is the meta :


    \textsc{Maclan pentagon is some recursive square !!} This recursive square \coqdocvar{normalize\_map\_assoc} is the ``functorial'' parallel to the normalization/flattening of binary trees; and is simply the unit of the reflection for the adjunction.


    The associative coherence comes before anything else, before the semiassociative completeness and before the semiassociative coherence :
\\  * The associative coherence, by the recursive square lemma, critically reduce to the classification of the endomorphisms in the semiassociative category. 
\\ * This associative coherence do not lack some ``Newman-style'' diamon lemma. The comonadic  adjunction, embedding : \coqdocvar{arrows} $\leftrightarrows$ \coqdocvar{arrows\_assoc} : flattening reflection, which says/subgeres that semiassociative coherence covers (and temporarily comes before) associative coherence is actually done posterior-ly (for want of formality), after it is already known that the simpler \coqdocvar{List} subcategory of \coqdocvar{arrows} made of endomorphisms is enough to cover (and is equivalent to) \coqdocvar{arrows\_assoc}.
\\ * The semiassociative coherence do lack some ``Newman-style'' diamon lemma; and, again posterior-ly, there exists some monadic adjunction, embedding : \coqdocvar{List} $\leftrightarrows$ \coqdocvar{arrows} : flattening reflection.


    The associative category is the meta of the semiassociative category, and the phrasing of semiassociative completeness \coqdocvar{lemma\_completeness} shows this actuality very clearly. The semiassociative coherence is done in some internal ( ``first/second order'' ?? ) encoding relative to associative coherence; may be exists some (yet to be found) ``higher-order'' / Coq Gallina encoding. The semiassociative coherence do lack some ``Newman-style'' diamon lemma. This diamon lemma is done in somme two-step process : first the codomain object of the diamon is assumed to be in normal/flattened form and this particularized diamon lemma \coqdocvar{lemma\_directedness} is proved without holding semiassociative completeness; then this assumption is erased and the full diamon lemma \coqdocvar{lemma\_coherence0} now necessitate the semiassociative completeness.


    Dosen book \cite{dosen} section 4.2 then section 4.3 is written into some computationally false order/precedence. The source of this falsification is the confusion/mixup between ``convertible'' (definitionality/meta equal) or ``propositional equal'' where the things are local variables instead of fully constructor-ed explicit terms.


    These Coq texts do not necessitate impredicativity and do not necessitate limitation of the flow of information from proof to data and do not necessitate large inductive types with polymorphic constructors and do not necessitate universe polymorphism; therefore no distinction between \coqdockw{Prop} or \coqdockw{Set} or \coqdockw{Type} is made in these Coq texts. Moreover this Coq texts do not exhitate to use do not hesitate to use the very sensible/fragile library \coqdocvar{Program.Equality} \coqdoctac{dependent} \coqdoctac{destruction} which may introduce extra non-necessary \coqdocvar{eq\_rect\_eq} axioms. It is most possible to erase this \coqdocvar{eq\_rect\_eq} ( coherence !! .. ) axiom from this Coq texts, otherwise this would be some coherence problem as deep/basic as associative coherence or semiassociative coherence.


    Noson Yanofsky talks about some Catalan categories to solve associative coherence in hes doctor text, may be the ``functorial'' normalization/flattening here is related to hes Catalan categories. Also the \coqdocvar{AAC} \coqdocvar{tactics} or \coqdocvar{CoqMT} which already come with COQ may be related, but the ultimate motivation is different ... \begin{coqdoccode}
\coqdocemptyline
\end{coqdoccode}
\vspace{-.15in} \coqdoceol
\coqdocemptyline
\coqdocnoindent
\coqdockw{Inductive} \coqdocvar{same\_assoc} : \coqdockw{\ensuremath{\forall}} \coqdocvar{A} \coqdocvar{B} : \coqdocvar{objects}, \coqdocvar{arrows\_assoc} \coqdocvar{A} \coqdocvar{B} \ensuremath{\rightarrow} \coqdocvar{arrows\_assoc} \coqdocvar{A} \coqdocvar{B} \ensuremath{\rightarrow} \coqdockw{Set}\coqdoceol
\coqdocindent{0.50em}
:= \coqdocvar{same\_assoc\_refl} : \coqdockw{\ensuremath{\forall}} (\coqdocvar{A} \coqdocvar{B} : \coqdocvar{objects}) (\coqdocvar{f} : \coqdocvar{arrows\_assoc} \coqdocvar{A} \coqdocvar{B}), \coqdocvar{f} \ensuremath{\sim}\coqdocvar{a} \coqdocvar{f}\coqdoceol
\coqdocindent{0.50em}
\ensuremath{|} ...\coqdoceol
\coqdocindent{0.50em}
\ensuremath{|} \coqdocvar{same\_assoc\_bracket\_left\_5} : \coqdockw{\ensuremath{\forall}} \coqdocvar{A} \coqdocvar{B} \coqdocvar{C} \coqdocvar{D} : \coqdocvar{objects},\coqdoceol
\coqdocindent{1.50em}
\coqdocvar{bracket\_left\_assoc} (\coqdocvar{A} /\symbol{92}0 \coqdocvar{B}) \coqdocvar{C} \coqdocvar{D} <\coqdocvar{oa} \coqdocvar{bracket\_left\_assoc} \coqdocvar{A} \coqdocvar{B} (\coqdocvar{C} /\symbol{92}0 \coqdocvar{D}) \ensuremath{\sim}\coqdocvar{a}\coqdoceol
\coqdocindent{1.50em}
\coqdocvar{bracket\_left\_assoc} \coqdocvar{A} \coqdocvar{B} \coqdocvar{C} /\symbol{92}1\coqdocvar{a} \coqdocvar{unitt\_assoc} \coqdocvar{D} <\coqdocvar{oa} \coqdocvar{bracket\_left\_assoc} \coqdocvar{A} (\coqdocvar{B} /\symbol{92}0 \coqdocvar{C}) \coqdocvar{D} <\coqdocvar{oa}\coqdoceol
\coqdocindent{1.50em}
\coqdocvar{unitt\_assoc} \coqdocvar{A} /\symbol{92}1\coqdocvar{a} \coqdocvar{bracket\_left\_assoc} \coqdocvar{B} \coqdocvar{C} \coqdocvar{D}.

\coqdocemptyline
\begin{coqdoccode}
\end{coqdoccode}
\vspace{-.15in} \coqdoceol
\coqdocemptyline
\coqdocnoindent
\coqdockw{Inductive} \coqdocvar{normal} : \coqdocvar{objects} \ensuremath{\rightarrow} \coqdockw{Set} :=\coqdoceol
\coqdocnoindent
\coqdocvar{normal\_cons1} : \coqdockw{\ensuremath{\forall}} \coqdocvar{l} : \coqdocvar{letters}, \coqdocvar{normal} (\coqdocvar{letter} \coqdocvar{l})\coqdoceol
\coqdocnoindent
\ensuremath{|} \coqdocvar{normal\_cons2} : \coqdockw{\ensuremath{\forall}} (\coqdocvar{A} : \coqdocvar{objects}) (\coqdocvar{l} : \coqdocvar{letters}), \coqdocvar{normal} \coqdocvar{A} \ensuremath{\rightarrow} \coqdocvar{normal} (\coqdocvar{A} /\symbol{92}0 \coqdocvar{letter} \coqdocvar{l}).

\coqdocemptyline
\begin{coqdoccode}
\end{coqdoccode}
\vspace{-.15in} \coqdoceol
\coqdocemptyline
\coqdocnoindent
\coqdockw{Fixpoint} \coqdocvar{normalize\_aux} (\coqdocvar{Z} \coqdocvar{A} : \coqdocvar{objects}) \{\coqdockw{struct} \coqdocvar{A}\} : \coqdocvar{objects} :=\coqdoceol
\coqdocindent{1.00em}
\coqdockw{match} \coqdocvar{A} \coqdockw{with}\coqdoceol
\coqdocindent{2.00em}
\ensuremath{|} \coqdocvar{letter} \coqdocvar{l} \ensuremath{\Rightarrow} \coqdocvar{Z} /\symbol{92}0 \coqdocvar{letter} \coqdocvar{l}\coqdoceol
\coqdocindent{2.00em}
\ensuremath{|} \coqdocvar{A1} /\symbol{92}0 \coqdocvar{A2} \ensuremath{\Rightarrow} (\coqdocvar{Z} </\symbol{92}0 \coqdocvar{A1}) </\symbol{92}0 \coqdocvar{A2}\coqdoceol
\coqdocindent{1.00em}
\coqdockw{end}\coqdoceol
\coqdocnoindent
\coqdockw{where} "Z </\symbol{92}0 A" := (\coqdocvar{normalize\_aux} \coqdocvar{Z} \coqdocvar{A}).

\coqdocemptyline
\begin{coqdoccode}
\end{coqdoccode}
\vspace{-.15in} \coqdoceol
\coqdocemptyline
\coqdocnoindent
\coqdockw{Fixpoint} \coqdocvar{normalize} (\coqdocvar{A} : \coqdocvar{objects}) : \coqdocvar{objects} :=\coqdoceol
\coqdocindent{1.00em}
\coqdockw{match} \coqdocvar{A} \coqdockw{with}\coqdoceol
\coqdocindent{2.00em}
\ensuremath{|} \coqdocvar{letter} \coqdocvar{l} \ensuremath{\Rightarrow} \coqdocvar{letter} \coqdocvar{l}\coqdoceol
\coqdocindent{2.00em}
\ensuremath{|} \coqdocvar{A1} /\symbol{92}0 \coqdocvar{A2} \ensuremath{\Rightarrow} (\coqdocvar{normalize} \coqdocvar{A1}) </\symbol{92}0 \coqdocvar{A2}\coqdoceol
\coqdocindent{1.00em}
\coqdockw{end}.

\coqdocemptyline
 Roughtly, the lemma \coqdocvar{development} takes as input some arrow term (bracket /\symbol{92} bracket) and output some \coqdocvar{developed} arrow term (bracket /\symbol{92} 1) o (1 /\symbol{92} bracket) which is \ensuremath{\sim}\coqdocvar{s} convertible (essentially by bifunctoriality of /\symbol{92}) to the input. Now surprisingly, this \coqdocvar{development} or factorization lemma necessitate some deep (`well-founded') induction, using some measure \coqdocvar{length} which shows that this may be related to arithmetic factorization. \begin{coqdoccode}
\coqdocemptyline
\end{coqdoccode}
\vspace{-.15in} \coqdoceol
\coqdocemptyline
\coqdocnoindent
\coqdockw{Fixpoint} \coqdocvar{length} (\coqdocvar{A} \coqdocvar{B} : \coqdocvar{objects}) (\coqdocvar{f} : \coqdocvar{arrows} \coqdocvar{A} \coqdocvar{B}) \{\coqdockw{struct} \coqdocvar{f}\} : \coqdocvar{nat} :=\coqdoceol
\coqdocindent{1.00em}
\coqdockw{match} \coqdocvar{f} \coqdockw{with}\coqdoceol
\coqdocindent{1.00em}
\ensuremath{|} \coqdocvar{unitt} \coqdocvar{\_} \ensuremath{\Rightarrow} 2\coqdoceol
\coqdocindent{1.00em}
\ensuremath{|} \coqdocvar{bracket\_left} \coqdocvar{\_} \coqdocvar{\_} \coqdocvar{\_} \ensuremath{\Rightarrow} 4\coqdoceol
\coqdocindent{1.00em}
\ensuremath{|} \coqdocvar{up\_1} \coqdocvar{A0} \coqdocvar{B0} \coqdocvar{A1} \coqdocvar{B1} \coqdocvar{f1} \coqdocvar{f2} \ensuremath{\Rightarrow} \coqdocvar{length} \coqdocvar{A0} \coqdocvar{B0} \coqdocvar{f1} \ensuremath{\times} \coqdocvar{length} \coqdocvar{A1} \coqdocvar{B1} \coqdocvar{f2}\coqdoceol
\coqdocindent{1.00em}
\ensuremath{|} \coqdocvar{com} \coqdocvar{A} \coqdocvar{B} \coqdocvar{C} \coqdocvar{f1} \coqdocvar{f2} \ensuremath{\Rightarrow} \coqdocvar{length} \coqdocvar{A} \coqdocvar{B} \coqdocvar{f1} + \coqdocvar{length} \coqdocvar{B} \coqdocvar{C} \coqdocvar{f2}\coqdoceol
\coqdocindent{1.00em}
\coqdockw{end}.

\coqdocemptyline
\begin{coqdoccode}
\end{coqdoccode}
\vspace{-.15in} \coqdoceol
\coqdocemptyline
\coqdocnoindent
\coqdockw{Lemma} \coqdocvar{development} : \coqdockw{\ensuremath{\forall}} (\coqdocvar{len} : \coqdocvar{nat}) (\coqdocvar{A} \coqdocvar{B} : \coqdocvar{objects}) (\coqdocvar{f} : \coqdocvar{arrows} \coqdocvar{A} \coqdocvar{B}),\coqdoceol
\coqdocindent{3.50em}
\coqdocvar{length} \coqdocvar{f} \ensuremath{\le} \coqdocvar{len} \ensuremath{\rightarrow} \{ \coqdocvar{f'} : \coqdocvar{arrows} \coqdocvar{A} \coqdocvar{B} \&\coqdoceol
\coqdocindent{17.00em}
\coqdocvar{developed} \coqdocvar{f'} \ensuremath{\times} ((\coqdocvar{length} \coqdocvar{f'} \ensuremath{\le} \coqdocvar{length} \coqdocvar{f}) \ensuremath{\times} (\coqdocvar{f} \ensuremath{\sim}\coqdocvar{s} \coqdocvar{f'})) \}.

\coqdocemptyline
\begin{coqdoccode}
\end{coqdoccode}
\vspace{-.15in} \coqdoceol
\coqdocemptyline
\coqdocnoindent
\coqdockw{Fixpoint} \coqdocvar{normalize\_aux\_unitrefl\_assoc} \coqdocvar{Y} \coqdocvar{Z} (\coqdocvar{y} : \coqdocvar{arrows\_assoc} \coqdocvar{Y} \coqdocvar{Z}) \coqdocvar{A} \coqdoceol
\coqdocindent{0.50em}
: \coqdocvar{arrows\_assoc} (\coqdocvar{Y} /\symbol{92}0 \coqdocvar{A}) (\coqdocvar{Z} </\symbol{92}0 \coqdocvar{A}) :=\coqdoceol
\coqdocindent{1.00em}
\coqdockw{match} \coqdocvar{A} \coqdockw{with}\coqdoceol
\coqdocindent{2.00em}
\ensuremath{|} \coqdocvar{letter} \coqdocvar{l} \ensuremath{\Rightarrow} \coqdocvar{y} /\symbol{92}1\coqdocvar{a} \coqdocvar{unitt\_assoc} (\coqdocvar{letter} \coqdocvar{l})\coqdoceol
\coqdocindent{2.00em}
\ensuremath{|} \coqdocvar{A1} /\symbol{92}0 \coqdocvar{A2} \ensuremath{\Rightarrow} ((\coqdocvar{y} </\symbol{92}1\coqdocvar{a} \coqdocvar{A1}) </\symbol{92}1\coqdocvar{a} \coqdocvar{A2}) <\coqdocvar{oa} \coqdocvar{bracket\_left\_assoc} \coqdocvar{Y} \coqdocvar{A1} \coqdocvar{A2}\coqdoceol
\coqdocindent{1.00em}
\coqdockw{end}\coqdoceol
\coqdocnoindent
\coqdockw{where} "y </\symbol{92}1a A" := (\coqdocvar{normalize\_aux\_unitrefl\_assoc} \coqdocvar{y} \coqdocvar{A}).

\coqdocemptyline
\begin{coqdoccode}
\end{coqdoccode}
\vspace{-.15in} \coqdoceol
\coqdocemptyline
\coqdocnoindent
\coqdockw{Fixpoint} \coqdocvar{normalize\_unitrefl\_assoc} (\coqdocvar{A} : \coqdocvar{objects}) : \coqdocvar{arrows\_assoc} \coqdocvar{A} (\coqdocvar{normalize} \coqdocvar{A}) :=\coqdoceol
\coqdocindent{1.00em}
\coqdockw{match} \coqdocvar{A} \coqdockw{with}\coqdoceol
\coqdocindent{2.00em}
\ensuremath{|} \coqdocvar{letter} \coqdocvar{l} \ensuremath{\Rightarrow} \coqdocvar{unitt\_assoc} (\coqdocvar{letter} \coqdocvar{l})\coqdoceol
\coqdocindent{2.00em}
\ensuremath{|} \coqdocvar{A1} /\symbol{92}0 \coqdocvar{A2} \ensuremath{\Rightarrow} (\coqdocvar{normalize\_unitrefl\_assoc} \coqdocvar{A1}) </\symbol{92}1\coqdocvar{a} \coqdocvar{A2}\coqdoceol
\coqdocindent{1.00em}
\coqdockw{end}.

\coqdocemptyline
\begin{coqdoccode}
\coqdocnoindent
\coqdockw{Check} \coqdoclemma{th151} : \coqdockw{\ensuremath{\forall}} \coqdocvar{A} : \coqdocinductive{objects}, \coqdocinductive{normal} \coqdocvariable{A} \ensuremath{\rightarrow} \coqdocdefinition{normalize} \coqdocvariable{A} \coqdocnotation{=} \coqdocvariable{A}.\coqdoceol
\end{coqdoccode}
Aborted th270 : for local variable \coqdocvar{A} with \coqdocvar{normal} \coqdocvar{A},
although there is the propositional equality \coqdocvar{th151}: \coqdocvar{normalize} \coqdocvar{A} = \coqdocvar{A}, one gets 
that \coqdocvar{normalize} \coqdocvar{A} and \coqdocvar{A} are not convertible (definitionally/meta equal);
therefore one shall not regard \coqdocvar{normalize\_unitrefl\_assoc} and \coqdocvar{unitt} \coqdocvar{A} as sharing
the same domain-codomain indices of \coqdocvar{arrows\_assoc}. \begin{coqdoccode}
\coqdocnoindent
\coqdockw{Check} \coqdoclemma{th260} : \coqdockw{\ensuremath{\forall}} \coqdocvar{N} \coqdocvar{P} : \coqdocinductive{objects}, \coqdocinductive{arrows\_assoc} \coqdocvariable{N} \coqdocvariable{P} \ensuremath{\rightarrow} \coqdocdefinition{normalize} \coqdocvariable{N} \coqdocnotation{=} \coqdocdefinition{normalize} \coqdocvariable{P}.\coqdoceol
\end{coqdoccode}
Aborted lemma\_coherence\_assoc0 : for local variables \coqdocvar{N}, \coqdocvar{P} with \coqdocvar{arrows\_assoc} \coqdocvar{N} \coqdocvar{P},
although there is the propositional equality \coqdocvar{th260} : \coqdocvar{normalize} \coqdocvar{N} = \coqdocvar{normalize} \coqdocvar{P},
one gets that \coqdocvar{normalize} \coqdocvar{A} and \coqdocvar{normalize} \coqdocvar{B} are not convertible (definitionally/meta equal);
therefore some transport other than \coqdocvar{eq\_rect}, some coherent transport is lacked. 

 Below \coqdocvar{directed} \coqdocvar{y} signify that \coqdocvar{y} is in the image of the embedding of \coqdocvar{arrows} into \coqdocvar{arrows\_assoc}. \begin{coqdoccode}
\coqdocemptyline
\coqdocnoindent
\coqdockw{Check} \coqdocdefinition{normalize\_aux\_map\_assoc} : \coqdockw{\ensuremath{\forall}} (\coqdocvar{X} \coqdocvar{Y} : \coqdocinductive{objects}) (\coqdocvar{x} : \coqdocinductive{arrows\_assoc} \coqdocvariable{X} \coqdocvariable{Y})\coqdoceol
\coqdocindent{4.50em}
(\coqdocvar{Z} : \coqdocinductive{objects}) (\coqdocvar{y} : \coqdocinductive{arrows\_assoc} \coqdocvariable{Y} \coqdocvariable{Z}), \coqdocinductive{directed} \coqdocvariable{y} \ensuremath{\rightarrow}\coqdoceol
\coqdocindent{3.50em}
\coqdockw{\ensuremath{\forall}} (\coqdocvar{A} \coqdocvar{B} : \coqdocinductive{objects}) (\coqdocvar{f} : \coqdocinductive{arrows\_assoc} \coqdocvariable{A} \coqdocvariable{B}),\coqdoceol
\coqdocindent{3.50em}
\coqdocnotation{\{} \coqdocvar{y\_map} \coqdocnotation{:} \coqdocinductive{arrows\_assoc} (\coqdocvariable{Y} \coqdocnotation{</\symbol{92}0} \coqdocvariable{A}) (\coqdocvariable{Z} \coqdocnotation{</\symbol{92}0} \coqdocvariable{B}) \coqdocnotation{\&}\coqdoceol
\coqdocindent{3.50em}
\coqdocnotation{(}\coqdocvar{y\_map} \coqdocnotation{<}\coqdocnotation{oa} \coqdocvariable{x} \coqref{itp2015.::x '</x5C1a' x}{\coqdocnotation{</\symbol{92}1}}\coqref{itp2015.::x '</x5C1a' x}{\coqdocnotation{a}} \coqdocvariable{A} \coqref{itp2015.::x 'x7Ea' x}{\coqdocnotation{\ensuremath{\sim}}}\coqref{itp2015.::x 'x7Ea' x}{\coqdocnotation{a}} \coqdocvariable{y} \coqref{itp2015.::x '</x5C1a' x}{\coqdocnotation{</\symbol{92}1}}\coqref{itp2015.::x '</x5C1a' x}{\coqdocnotation{a}} \coqdocvariable{B} \coqdocnotation{<}\coqdocnotation{oa} \coqdocvariable{x} \coqdocnotation{/\symbol{92}1}\coqdocnotation{a} \coqdocvariable{f}\coqdocnotation{)} \coqdocnotation{\ensuremath{\times}} \coqdocinductive{directed} \coqdocvar{y\_map} \coqdocnotation{\}}.\coqdoceol
\coqdocemptyline
\coqdocnoindent
\coqdockw{Check} \coqdocdefinition{normalize\_map\_assoc} : \coqdockw{\ensuremath{\forall}} (\coqdocvar{A} \coqdocvar{B} : \coqdocinductive{objects}) (\coqdocvar{f} : \coqdocinductive{arrows\_assoc} \coqdocvariable{A} \coqdocvariable{B}),\coqdoceol
\coqdocindent{3.50em}
\coqdocnotation{\{} \coqdocvar{y\_map} \coqdocnotation{:} \coqdocinductive{arrows\_assoc} (\coqdocdefinition{normalize} \coqdocvariable{A}) (\coqdocdefinition{normalize} \coqdocvariable{B}) \coqdocnotation{\&}\coqdoceol
\coqdocindent{3.50em}
\coqdocnotation{(}\coqdocvar{y\_map} \coqdocnotation{<}\coqdocnotation{oa} \coqref{itp2015.normalize unitrefl assoc}{\coqdocabbreviation{normalize\_unitrefl\_assoc}} \coqdocvariable{A} \coqref{itp2015.::x 'x7Ea' x}{\coqdocnotation{\ensuremath{\sim}}}\coqref{itp2015.::x 'x7Ea' x}{\coqdocnotation{a}}\coqdoceol
\coqdocindent{4.50em}
\coqref{itp2015.normalize unitrefl assoc}{\coqdocabbreviation{normalize\_unitrefl\_assoc}} \coqdocvariable{B} \coqdocnotation{<}\coqdocnotation{oa} \coqdocvariable{f}\coqdocnotation{)} \coqdocnotation{\ensuremath{\times}} \coqdocinductive{directed} \coqdocvar{y\_map} \coqdocnotation{\}}.\coqdoceol
\coqdocemptyline
\end{coqdoccode}
\section{Dosen SemiAssociative Coherence}

\begin{coqdoccode}
\end{coqdoccode}
\vspace{-.15in} \coqdoceol
\coqdocemptyline
\coqdocnoindent
\coqdockw{Inductive} \coqdocvar{nodes} : \coqdocvar{objects} \ensuremath{\rightarrow} \coqdockw{Set} :=\coqdoceol
\coqdocindent{2.00em}
\coqdocvar{self} : \coqdockw{\ensuremath{\forall}} \coqdocvar{A} : \coqdocvar{objects}, \coqdocvar{A}\coqdoceol
\coqdocindent{1.00em}
\ensuremath{|} \coqdocvar{at\_left} : \coqdockw{\ensuremath{\forall}} \coqdocvar{A} : \coqdocvar{objects}, \coqdocvar{A} \ensuremath{\rightarrow} \coqdockw{\ensuremath{\forall}} \coqdocvar{B} : \coqdocvar{objects}, \coqdocvar{A} /\symbol{92}0 \coqdocvar{B}\coqdoceol
\coqdocindent{1.00em}
\ensuremath{|} \coqdocvar{at\_right} : \coqdockw{\ensuremath{\forall}} \coqdocvar{A} \coqdocvar{B} : \coqdocvar{objects}, \coqdocvar{B} \ensuremath{\rightarrow} \coqdocvar{A} /\symbol{92}0 \coqdocvar{B}.

\coqdocemptyline
 \begin{coqdoccode}
\end{coqdoccode}
\vspace{-.15in} \coqdoceol
\coqdocemptyline
\coqdocnoindent
\coqdockw{Inductive} \coqdocvar{lt\_right} : \coqdockw{\ensuremath{\forall}} \coqdocvar{A} : \coqdocvar{objects}, \coqdocvar{A} \ensuremath{\rightarrow} \coqdocvar{A} \ensuremath{\rightarrow} \coqdockw{Set} :=\coqdoceol
\coqdocindent{2.00em}
\coqdocvar{lt\_right\_cons1} : \coqdockw{\ensuremath{\forall}} (\coqdocvar{B} : \coqdocvar{objects}) (\coqdocvar{z} : \coqdocvar{B}) (\coqdocvar{C} : \coqdocvar{objects}),\coqdoceol
\coqdocindent{10.50em}
\coqdocvar{self} (\coqdocvar{C} /\symbol{92}0 \coqdocvar{B}) <\coqdocvar{r} \coqdocvar{at\_right} \coqdocvar{C} \coqdocvar{z}\coqdoceol
\coqdocindent{1.00em}
\ensuremath{|} \coqdocvar{lt\_right\_cons2} : \coqdockw{\ensuremath{\forall}} (\coqdocvar{B} \coqdocvar{C} : \coqdocvar{objects}) (\coqdocvar{x} \coqdocvar{y} : \coqdocvar{B}),\coqdoceol
\coqdocindent{10.50em}
\coqdocvar{x} <\coqdocvar{r} \coqdocvar{y} \ensuremath{\rightarrow} \coqdocvar{at\_left} \coqdocvar{x} \coqdocvar{C} <\coqdocvar{r} \coqdocvar{at\_left} \coqdocvar{y} \coqdocvar{C}\coqdoceol
\coqdocindent{1.00em}
\ensuremath{|} \coqdocvar{lt\_right\_cons3} : \coqdockw{\ensuremath{\forall}} (\coqdocvar{B} \coqdocvar{C} : \coqdocvar{objects}) (\coqdocvar{x} \coqdocvar{y} : \coqdocvar{B}),\coqdoceol
\coqdocindent{10.50em}
\coqdocvar{x} <\coqdocvar{r} \coqdocvar{y} \ensuremath{\rightarrow} \coqdocvar{at\_right} \coqdocvar{C} \coqdocvar{x} <\coqdocvar{r} \coqdocvar{at\_right} \coqdocvar{C} \coqdocvar{y}.

\coqdocemptyline
\begin{coqdoccode}
\end{coqdoccode}
\vspace{-.15in} \coqdoceol
\coqdocemptyline
\coqdocnoindent
\coqdockw{Definition} \coqdocvar{bracket\_left\_on\_nodes} (\coqdocvar{A} \coqdocvar{B} \coqdocvar{C} : \coqdocvar{objects})\coqdoceol
\coqdocindent{0.50em}
( \coqdocvar{x} : \coqdocvar{nodes} (\coqdocvar{A} /\symbol{92}0 (\coqdocvar{B} /\symbol{92}0 \coqdocvar{C})) ) : \coqdocvar{nodes} ((\coqdocvar{A} /\symbol{92}0 \coqdocvar{B}) /\symbol{92}0 \coqdocvar{C}).\coqdoceol
\coqdocindent{2.00em}
\coqdoctac{dependent} \coqdoctac{destruction} \coqdocvar{x}.\coqdoceol
\coqdocindent{4.00em}
\coqdoctac{exact} (\coqdocvar{at\_left} (\coqdocvar{self} (\coqdocvar{A} /\symbol{92}0 \coqdocvar{B})) \coqdocvar{C}).\coqdoceol
\coqdocindent{4.00em}
\coqdoctac{exact} (\coqdocvar{at\_left} (\coqdocvar{at\_left} \coqdocvar{x} \coqdocvar{B}) \coqdocvar{C}).\coqdoceol
\coqdocindent{4.00em}
\coqdoctac{dependent} \coqdoctac{destruction} \coqdocvar{x}.\coqdoceol
\coqdocindent{6.00em}
\coqdoctac{exact} (\coqdocvar{self} ((\coqdocvar{A} /\symbol{92}0 \coqdocvar{B}) /\symbol{92}0 \coqdocvar{C})).\coqdoceol
\coqdocindent{6.00em}
\coqdoctac{exact} (\coqdocvar{at\_left} (\coqdocvar{at\_right} \coqdocvar{A} \coqdocvar{x}) \coqdocvar{C}).\coqdoceol
\coqdocindent{6.00em}
\coqdoctac{exact} (\coqdocvar{at\_right} (\coqdocvar{A} /\symbol{92}0 \coqdocvar{B}) \coqdocvar{x}). \coqdockw{Defined}.

\coqdocemptyline
\begin{coqdoccode}
\coqdocnoindent
\coqdockw{Check} \coqdocdefinition{arrows\_assoc\_on\_nodes} : \coqdockw{\ensuremath{\forall}} \coqdocvar{A} \coqdocvar{B}, \coqdocinductive{arrows\_assoc} \coqdocvariable{A} \coqdocvariable{B} \ensuremath{\rightarrow} \coqdocinductive{nodes} \coqdocvariable{A} \ensuremath{\rightarrow} \coqdocinductive{nodes} \coqdocvariable{B}.\coqdoceol
\coqdocemptyline
\end{coqdoccode}
Soundness : (using \coqdocvar{nodes} and \coqdocvar{arrows\_assoc\_on\_nodes} notations coercions) \begin{coqdoccode}
\coqdocemptyline
\coqdocnoindent
\coqdockw{Check} \coqref{itp2015.lemma soundness}{\coqdocabbreviation{lemma\_soundness}} : \coqdockw{\ensuremath{\forall}} \coqdocvar{A} \coqdocvar{B} (\coqdocvar{f} : \coqdocinductive{arrows} \coqdocvariable{A} \coqdocvariable{B}) (\coqdocvar{x} \coqdocvar{y} : \coqdocvariable{A}), \coqdocvariable{f} \coqdocvariable{x} \coqref{itp2015.::x '<r' x}{\coqdocnotation{<}}\coqref{itp2015.::x '<r' x}{\coqdocnotation{r}} \coqdocvariable{f} \coqdocvariable{y} \ensuremath{\rightarrow} \coqdocvariable{x} \coqref{itp2015.::x '<r' x}{\coqdocnotation{<}}\coqref{itp2015.::x '<r' x}{\coqdocnotation{r}} \coqdocvariable{y}.\coqdoceol
\coqdocemptyline
\end{coqdoccode}
Completeness : lemma \coqdocvar{lem005700} is deep (`well-founded') induction on \coqdocvar{lengthn'{}'}, with accumulator/continuation \coqdocvar{cumul\_letteries}. The prerequisites are 4000 lines of multifolded/complicated Coq text, and some of which are listed below : \begin{coqdoccode}
\coqdocemptyline
\coqdocnoindent
\coqdockw{Check} \coqdoclemma{lemma\_completeness} : \coqdockw{\ensuremath{\forall}} (\coqdocvar{B} \coqdocvar{A} : \coqdocinductive{objects}) (\coqdocvar{f} : \coqdocinductive{arrows\_assoc} \coqdocvariable{B} \coqdocvariable{A}),\coqdoceol
\coqdocindent{4.50em}
(\coqdockw{\ensuremath{\forall}} \coqdocvar{x} \coqdocvar{y} : \coqdocinductive{nodes} \coqdocvariable{B}, \coqdocvariable{x} \coqref{itp2015.::x '<r' x}{\coqdocnotation{<}}\coqref{itp2015.::x '<r' x}{\coqdocnotation{r}} \coqdocvariable{y} \ensuremath{\rightarrow} \coqref{itp2015.::x '<r' x}{\coqdocnotation{(}}\coqdocvariable{f} \coqdocvariable{x}\coqref{itp2015.::x '<r' x}{\coqdocnotation{)}} \coqref{itp2015.::x '<r' x}{\coqdocnotation{<}}\coqref{itp2015.::x '<r' x}{\coqdocnotation{r}} \coqref{itp2015.::x '<r' x}{\coqdocnotation{(}}\coqdocvariable{f} \coqdocvariable{y}\coqref{itp2015.::x '<r' x}{\coqdocnotation{)}}) \ensuremath{\rightarrow} \coqdocinductive{arrows} \coqdocvariable{A} \coqdocvariable{B}.\coqdoceol
\coqdocemptyline
\coqdocnoindent
\coqdockw{Check} \coqdoclemma{lem005700} : \coqdockw{\ensuremath{\forall}} (\coqdocvar{B} : \coqdocinductive{objects}) (\coqdocvar{len} : \coqdocinductive{nat}),\coqdoceol
\coqdocindent{0.50em}
\coqdockw{\ensuremath{\forall}} (\coqdocvar{cumul\_letteries} : \coqdocinductive{nodes} \coqdocvariable{B} \ensuremath{\rightarrow} \coqdocinductive{bool})\coqdoceol
\coqdocindent{0.50em}
(\coqdocvar{H\_cumul\_letteries\_wellform} : \coqdocabbreviation{cumul\_letteries\_wellform'} \coqdocvariable{B} \coqdocvariable{cumul\_letteries})\coqdoceol
\coqdocindent{0.50em}
(\coqdocvar{H\_cumul\_letteries\_satur} : \coqdockw{\ensuremath{\forall}} \coqdocvar{y} : \coqdocinductive{nodes} \coqdocvariable{B}, \coqdocvariable{cumul\_letteries} \coqdocvariable{y} \coqdocnotation{=} \coqdocconstructor{true}\coqdoceol
\coqdocindent{2.50em}
\ensuremath{\rightarrow} \coqdockw{\ensuremath{\forall}} \coqdocvar{z} : \coqdocinductive{nodes} \coqdocvariable{B}, \coqdocabbreviation{lt\_leftorright\_eq} \coqdocvariable{y} \coqdocvariable{z} \ensuremath{\rightarrow} \coqdocvariable{cumul\_letteries} \coqdocvariable{z} \coqdocnotation{=} \coqdocconstructor{true})\coqdoceol
\coqdocindent{0.50em}
(\coqdocvar{H\_len} : \coqdocdefinition{lengthn'{}'} \coqdocvariable{cumul\_letteries} \coqdocvariable{H\_cumul\_letteries\_wellform} \coqdocnotation{\ensuremath{\le}} \coqdocvariable{len}),\coqdoceol
\coqdocindent{0.50em}
\coqdockw{\ensuremath{\forall}} (\coqdocvar{A} : \coqdocinductive{objects}) (\coqdocvar{f} : \coqdocinductive{arrows\_assoc} \coqdocvariable{B} \coqdocvariable{A})\coqdoceol
\coqdocindent{0.50em}
(\coqdocvar{H\_node\_is\_lettery} : \coqdockw{\ensuremath{\forall}} \coqdocvar{x} : \coqdocinductive{nodes} \coqdocvariable{B}, \coqdocvariable{cumul\_letteries} \coqdocvariable{x} \coqdocnotation{=} \coqdocconstructor{true} \ensuremath{\rightarrow} \coqdocabbreviation{node\_is\_lettery} \coqdocvariable{f} \coqdocvariable{x})\coqdoceol
\coqdocindent{0.50em}
(\coqdocvar{H\_object\_at\_node} : \coqdockw{\ensuremath{\forall}} \coqdocvar{x} : \coqdocinductive{nodes} \coqdocvariable{B}, \coqdocvariable{cumul\_letteries} \coqdocvariable{x} \coqdocnotation{=} \coqdocconstructor{true} \coqdoceol
\coqdocindent{2.50em}
\ensuremath{\rightarrow} \coqdocdefinition{object\_at\_node} \coqdocvariable{x} \coqdocnotation{=} \coqdocdefinition{object\_at\_node} (\coqdocvariable{f} \coqdocvariable{x}))\coqdoceol
\coqdocindent{0.50em}
(\coqdocvar{H\_cumul\_B} : \coqdockw{\ensuremath{\forall}} \coqdocvar{x} \coqdocvar{y} : \coqdocinductive{nodes} \coqdocvariable{B}, \coqdocvariable{x} \coqref{itp2015.::x '<r' x}{\coqdocnotation{<}}\coqref{itp2015.::x '<r' x}{\coqdocnotation{r}} \coqdocvariable{y} \ensuremath{\rightarrow} \coqref{itp2015.::x '<r' x}{\coqdocnotation{(}}\coqdocvariable{f} \coqdocvariable{x}\coqref{itp2015.::x '<r' x}{\coqdocnotation{)}} \coqref{itp2015.::x '<r' x}{\coqdocnotation{<}}\coqref{itp2015.::x '<r' x}{\coqdocnotation{r}} \coqref{itp2015.::x '<r' x}{\coqdocnotation{(}}\coqdocvariable{f} \coqdocvariable{y}\coqref{itp2015.::x '<r' x}{\coqdocnotation{)}}),    \coqdocinductive{arrows} \coqdocvariable{A} \coqdocvariable{B}.\coqdoceol
\end{coqdoccode}
\vspace{-.15in} \coqdoceol
\coqdocemptyline
\coqdocnoindent
\coqdockw{Notation} \coqdocvar{lt\_leftorright\_eq} \coqdocvar{x} \coqdocvar{y} := (\coqdocvar{sum} (\coqdocvar{eq} \coqdocvar{x} \coqdocvar{y}) (\coqdocvar{sum} (\coqdocvar{x} <\coqdocvar{l} \coqdocvar{y}) (\coqdocvar{x} <\coqdocvar{r} \coqdocvar{y}))).

\coqdocemptyline
 And \coqdocvar{nodal\_multi\_bracket\_left\_full}, which is some localized/deep \coqdocvar{multi\_bracket\_left} at some internal node, is one of the most complicated/multifolded construction in this Coq text. And \coqdocvar{nodal\_multi\_bracket\_left\_full} below and later really lack this constructive equality \coqdocvar{objects\_same}, so that we get some transport map which is coherent, transport map other than \coqdocvar{eq\_rect}. Maybe it is possible to effectively  use the constructive equality \coqdocvar{objects\_same} at more places instead of \coqdocvar{eq}. \begin{coqdoccode}
\coqdocemptyline
\end{coqdoccode}
\vspace{-.15in} \coqdoceol
\coqdocemptyline
\coqdocnoindent
\coqdockw{Inductive} \coqdocvar{objects\_same} : \coqdocvar{objects} \ensuremath{\rightarrow} \coqdocvar{objects} \ensuremath{\rightarrow} \coqdockw{Set} :=\coqdoceol
\coqdocindent{2.00em}
\coqdocvar{objects\_same\_cons1} : \coqdockw{\ensuremath{\forall}} \coqdocvar{l} : \coqdocvar{letters}, \coqdocvar{objects\_same} (\coqdocvar{letter} \coqdocvar{l}) (\coqdocvar{letter} \coqdocvar{l})\coqdoceol
\coqdocindent{1.00em}
\ensuremath{|} \coqdocvar{objects\_same\_cons2} : \coqdockw{\ensuremath{\forall}} \coqdocvar{A} \coqdocvar{A'} : \coqdocvar{objects}, \coqdocvar{objects\_same} \coqdocvar{A} \coqdocvar{A'} \ensuremath{\rightarrow}\coqdoceol
\coqdocindent{12.50em}
\coqdockw{\ensuremath{\forall}} \coqdocvar{B} \coqdocvar{B'} : \coqdocvar{objects}, \coqdocvar{objects\_same} \coqdocvar{B} \coqdocvar{B'} \ensuremath{\rightarrow}\coqdoceol
\coqdocindent{12.50em}
\coqdocvar{objects\_same} (\coqdocvar{A} /\symbol{92}0 \coqdocvar{B}) (\coqdocvar{A'} /\symbol{92}0 \coqdocvar{B'}).

\coqdocemptyline
\begin{coqdoccode}
\end{coqdoccode}
\vspace{-.15in} \coqdoceol
\coqdocemptyline
\coqdocnoindent
\coqdockw{Fixpoint} \coqdocvar{foldright} (\coqdocvar{A} : \coqdocvar{objects}) (\coqdocvar{Dlist} : \coqdocvar{list} \coqdocvar{objects}) \{\coqdockw{struct} \coqdocvar{Dlist}\} : \coqdocvar{objects} :=\coqdoceol
\coqdocindent{1.00em}
\coqdockw{match} \coqdocvar{Dlist} \coqdockw{with}\coqdoceol
\coqdocindent{2.00em}
\ensuremath{|} \coqdocvar{nil} \ensuremath{\Rightarrow} \coqdocvar{A}\coqdoceol
\coqdocindent{2.00em}
\ensuremath{|} \coqdocvar{D0} :: \coqdocvar{Dlist0} \ensuremath{\Rightarrow} (\coqdocvar{A} /\symbol{92}\symbol{92} \coqdocvar{Dlist0}) /\symbol{92}0 \coqdocvar{D0}\coqdoceol
\coqdocnoindent
\coqdockw{where} "A /\symbol{92}\symbol{92} Dlist" := (\coqdocvar{foldright} \coqdocvar{A} \coqdocvar{Dlist}).

\coqdocemptyline
\begin{coqdoccode}
\coqdocemptyline
\coqdocnoindent
\coqdockw{Check} \coqdocdefinition{multi\_bracket\_left} : \coqdockw{\ensuremath{\forall}} (\coqdocvar{A} \coqdocvar{B} \coqdocvar{C} : \coqdocinductive{objects}) (\coqdocvar{Dlist} : \coqdocinductive{list} \coqdocinductive{objects}),\coqdoceol
\coqdocindent{3.50em}
\coqdocinductive{arrows} (\coqdocvariable{A} \coqdocnotation{/\symbol{92}0} \coqdocnotation{(}\coqdocvariable{B} \coqdocnotation{/\symbol{92}0} \coqdocvariable{C} \coqdocnotation{/\symbol{92}\symbol{92}} \coqdocvariable{Dlist}\coqdocnotation{)}) (\coqdocnotation{(}\coqdocvariable{A} \coqdocnotation{/\symbol{92}0} \coqdocvariable{B}\coqdocnotation{)} \coqdocnotation{/\symbol{92}0} \coqdocvariable{C} \coqdocnotation{/\symbol{92}\symbol{92}} \coqdocvariable{Dlist}).\coqdoceol
\end{coqdoccode}
 \vspace{-.15in} \coqdoceol
\coqdocemptyline
\coqdocnoindent
\coqdockw{Fixpoint} \coqdocvar{object\_at\_node} (\coqdocvar{A} : \coqdocvar{objects}) (\coqdocvar{x} : \coqdocvar{A}) \{\coqdockw{struct} \coqdocvar{x}\} : \coqdocvar{objects} :=\coqdoceol
\coqdocindent{1.00em}
\coqdockw{match} \coqdocvar{x} \coqdockw{with}\coqdoceol
\coqdocindent{1.00em}
\ensuremath{|} \coqdocvar{self} \coqdocvar{A0} \ensuremath{\Rightarrow} \coqdocvar{A0}\coqdoceol
\coqdocindent{1.00em}
\ensuremath{|} \coqdocvar{at\_left} \coqdocvar{A0} \coqdocvar{x0} \coqdocvar{\_} \ensuremath{\Rightarrow} \coqdocvar{object\_at\_node} \coqdocvar{A0} \coqdocvar{x0}\coqdoceol
\coqdocindent{1.00em}
\ensuremath{|} \coqdocvar{at\_right} \coqdocvar{\_} \coqdocvar{B} \coqdocvar{x0} \ensuremath{\Rightarrow} \coqdocvar{object\_at\_node} \coqdocvar{B} \coqdocvar{x0}\coqdoceol
\coqdocindent{1.00em}
\coqdockw{end}.

\coqdocemptyline
\begin{coqdoccode}
\end{coqdoccode}
\vspace{-.15in} \coqdoceol
\coqdocemptyline
\coqdocnoindent
\coqdockw{Inductive} \coqdocvar{object\_is\_letter} : \coqdocvar{objects} \ensuremath{\rightarrow} \coqdockw{Set} :=\coqdoceol
\coqdocindent{2.00em}
\coqdocvar{object\_is\_letter\_cons} : \coqdockw{\ensuremath{\forall}} \coqdocvar{l} : \coqdocvar{letters}, \coqdocvar{object\_is\_letter} (\coqdocvar{letter} \coqdocvar{l}).

\coqdocemptyline
\begin{coqdoccode}
\end{coqdoccode}
\vspace{-.15in} \coqdoceol
\coqdocemptyline
\coqdocnoindent
\coqdockw{Notation} \coqdocvar{node\_is\_letter} \coqdocvar{x} := (\coqdocvar{object\_is\_letter} (\coqdocvar{object\_at\_node} \coqdocvar{x})).

\coqdocemptyline
\begin{coqdoccode}
\end{coqdoccode}
\vspace{-.15in} \coqdoceol
\coqdocemptyline
\coqdocnoindent
\coqdockw{Notation} \coqdocvar{node\_is\_lettery} \coqdocvar{f} \coqdocvar{w} := (\coqdocvar{prod}\coqdoceol
\coqdocnoindent
( \coqdockw{\ensuremath{\forall}} (\coqdocvar{x} : \coqdocvar{nodes} \coqdocvar{\_}), \coqdocvar{lt\_leftorright\_eq} \coqdocvar{w} \coqdocvar{x} \ensuremath{\rightarrow} \coqdocvar{lt\_leftorright\_eq} (\coqdocvar{f} \coqdocvar{w}) (\coqdocvar{f} \coqdocvar{x}) )\coqdoceol
\coqdocnoindent
( \coqdockw{\ensuremath{\forall}} (\coqdocvar{x} : \coqdocvar{nodes} \coqdocvar{\_}), \coqdocvar{lt\_leftorright\_eq} (\coqdocvar{f} \coqdocvar{w}) (\coqdocvar{f} \coqdocvar{x}) \coqdoceol
\coqdocindent{3.00em}
\ensuremath{\rightarrow} \coqdocvar{lt\_leftorright\_eq} ((\coqdocvar{rev} \coqdocvar{f}) (\coqdocvar{f} \coqdocvar{w})) ((\coqdocvar{rev} \coqdocvar{f}) (\coqdocvar{f} \coqdocvar{x})) )).

\coqdocemptyline
\begin{coqdoccode}
\end{coqdoccode}
\vspace{-.15in} \coqdoceol
\coqdocemptyline
\coqdocnoindent
\coqdockw{Notation} \coqdocvar{cumul\_letteries\_wellform'} \coqdocvar{B} \coqdocvar{cumul\_letteries} :=\coqdoceol
\coqdocindent{1.00em}
(\coqdockw{\ensuremath{\forall}} \coqdocvar{x} : \coqdocvar{B}, \coqdocvar{object\_is\_letter} (\coqdocvar{object\_at\_node} \coqdocvar{x}) \ensuremath{\rightarrow} \coqdocvar{eq} (\coqdocvar{cumul\_letteries} \coqdocvar{x}) \coqdocvar{true}).

\coqdocemptyline
\begin{coqdoccode}
\end{coqdoccode}

\begin{thebibliography}{1}
\bibitem{gross}Jason Gross, Adam Chlipala, David I. Spivak. “Experience
Implementing a Performant Category-Theory Library in Coq” In: Interactive
Theorem Proving. Springer, 2014.

\bibitem{chlipala}Gregory Malecha, Adam Chlipala, Thomas Braibant.
“Compositional Computational Reflection” In: Interactive Theorem Proving.
Springer, 2014.

\bibitem{chlipalacpdt}Adam Chlipala. ``Certified Programming with
Dependent Types''\\
\url{http://adam.chlipala.net/cpdt/}

\bibitem{dosen}Kosta Dosen, Zoran Petric. “Proof-Theoretical Coherence”
\\
\url{http://www.mi.sanu.ac.rs/~kosta/coh.pdf} , 2007.

\bibitem{aigner}Martin Aigner. ``Combinatorial Theory'' Springer,
1997

\bibitem{borceuxjanelidze}Francis Borceux, George Janelidze. “Galois
Theories” Cambridge University Press, 2001.

\bibitem{borceux}Francis Borceux. “Handbook of categorical algebra.
Volumes 1 2 3” Cambridge University Press, 1994.\end{thebibliography}

\end{document}
